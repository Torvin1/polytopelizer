\documentclass[11pt]{article}    
\usepackage[ngerman]{babel}
\usepackage{ngerman}

\begin{document}
\section{Ausgabe}
\subsection{Zielsetzung}
Nach dem Durchlaufen des Algorithmus sollten verschiedene Ausgabem"oglichkeiten implementiert werden:
\begin{itemize}
\item[1.] 3-dimensionales drehbares Polytope
\item[2.] Sicherung als .jpg
\item[3.] Auflistung der Eckpunkte
\end{itemize}
\subsection{Umsetzung}
F"ur die grafischen Darstellungen der Polytope haben wir uns f"ur die Java-Bibliothek jReality entschieden. Diese wird von der TU-Berlin entwickelt und steht unter der BSD-Lizenz.
jReality bietet die M"oglichkeit, einen bereits voll funktionsf"ahigen Viewer zu nutzen. Dieser hat f"ur unsere Zwecke allerdings zu viele Funktionen. Alternativ ist es m"oglich nur eine Komponente zu erstellen, die das Rendering "ubernimmt. Au"serdem k"onnen dieser Komponente \"Tools\" hinzugef"ugt werden, die es z.B. erm"oglichen das dargestellte 3 dimensionale Objekt zu drehen. Diese Komponente ist genauso wie AWT- oder Swing-Komponenten nutzbar, kann also beliebig, z.B. auf einem Frame, angezeigt werden.
\subsection{Ergebnis}
Unsere Software erm"oglicht die Darstellung des berechneten Polytops als 3 dimensionales Objekt. Diese kann beliebig gedreht und vergr"o"sert werden. Eine Auflistung der Eckpunkte des Polytopes ist per Buttonklick entweder als .pol Datei abspeicherbar oder kann in einem extra Fenster eingesehen werden.
Speichern der momentanen Sicht auf das Polytope als .jpg ist ebenfals m"oglich.
\subsection{Probleme}
Aufgrund der sehr sp"arlichen Dokumentation von jReality war es schwierig herauszufinden, welche M"oglichkeiten diese Bibliothek bietet. Durch den vordesignten Viewer war allerdings schnell zu erkennen, dass sie alle Funktionen, die f"ur unser Projekt ben"otigt wurden, bereitstellt. Wie man diese allerdings auf einer individuellen Benutzeroberfl"ache nutzt, war meist nur durch intensive Nutzung von Suchmaschinen oder durchsuchen des Codes von jReality selbst zu erkennen.

Bei der Darstellung des Polytopes gibt es allerdings bei einer gro"sen Anzahl von Punkten in dem eingegebenen Netzwerk ein Problem. Die Fl"achen werden teilweise so klein, dass sie bei der Darstellung in jReality nicht mehr zu erkennen sind. Dies spiegelt sich leider auch beim Speichern als .jpg wieder.
jReality bietet die M"oglichkeit, die momentane Sicht einer Anzeigekomponente als BufferedImage zu exportieren. Dieses kann dann mit der Java ImageIO in einem beliebeigen Format gespeichert werden. Dieses BufferedImage hat allerdings eine relative kleine Aufl"osung (hier 685x600). W"are es m"oglich, diese Aufl"osung zu vergr"o"sern, k"onnten in einem Bild mit hoher Auf"osung alle Punkte erkannt werden. Dies ist so nicht m"oglich.
\end{document}